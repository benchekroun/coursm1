\documentclass{article}
\usepackage[utf8]{inputenc} % un package
\usepackage[T1]{fontenc}      % un second package
\usepackage[francais]{babel}  % un troisième package
\usepackage{lmodern}
\usepackage{amsmath}
\usepackage{amssymb}
\usepackage{mathrsfs}


\title{Programmation graphique - R.jennane}
\author{Alexandre Masson}
\date{15 Janvier 2013}

\begin{document}
\maketitle
\newpage
\tableofcontents
\newpage
\section{Filtre linéaire}
\paragraph{}la décomposition d'un filtre en filtre plus petit, réduit le nombre d'opération par pixel, ce qui est remarquable pour des grandes images.\\\\Les filtres à noyau séparables peuvent être déployés autrement, par exemple un filtre 2D a noyau séparable pourra être appliqué ligne par ligne et colonne par colonne , soit en une dimension au lieu de deux.\\La transformée de fourier-1 ramène à la convolution d'une image avec des filtres gaussien.\\\\
Le chapeau mexicain , c'est le filtre gaussien.
\section{Filtre passe basse} Basse fréquence : variations lente, pas de brusque variation, contrairement aux hautes fréquences qui sont des variations violentes qui suivant la courbe de l'image. Bruit:  brusque variation d'un pixel par rapport à ses voisins. Le pixels central est toujours mis en avant , car son poids est plus important.
\paragraph{Filtre passe bas} on ne souhaite garder que les basses fréquence. la gaussienne permet de créer les matrices du filtre.\\\\Passe-tout= passehaut+passe bas.\\highboost
\paragraph{opérateurs dérivés} Contour  = transition brusque entre deux zone de même valeur, on le souhait nette, mais il est plutôt courbe, en vrai il a souvent du bruit, il est donc nécessaire de supprimer le bruit, puis de faire une passe haute et trouver un opérateur dérivé.dérivé première : gradient, dérivée seconde : laplacien . problème:  accentuation des high freq, augmentation d bruit au détriment des informations utiles, on peux la corriger avec les dérivés filtrés, remplacer le filtre -1,1 par -1,0,1, et on peux aussi appliquer un filtre lisseur. pour les dérivés en 2D, il existe déjà des filtres , sobel et prewitt.\\\\Laplacien : dérivée seconde, résultats indep de la detec des contours.
\section{Filtre non linéaire}\paragraph{filtre médian} Il n'est pas linéaire car pas de combinaisons de pixels,  mais plutôt choix du pixels médian parmi les pixels. plus le masque est grand, plus le filtre est efficace, mais plus il déforme l'image.
\paragraph{traitements globaux} tous les pixels interviennent dans la modification d'un pixels
\end{document}