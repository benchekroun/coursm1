\documentclass{article}
\usepackage[utf8]{inputenc} % un package
\usepackage[T1]{fontenc}      % un second package
\usepackage[francais]{babel}  % un troisième package


\title{Travaux d’Études et de Recherche - Jean-Michel Couvreur}
\author{Alexandre Masson}
\date{14 Janvier 2013}

\begin{document}
\maketitle
\newpage
\section{C'est quoi?} Le but principal c'est en plus de programmer un logiciel, c’est de transmettre, étudier , et synthétiser votre travail, il va nous être demander un travail de documentation et l'étude bibliographique. On attend aussi de la rigueur dans les affirmations, il faut être capable de justifier, faire des tests, utiliser des outils de vérifications.\\\\Ça regroupe des expériences de génie logiciel : développement en groupe autour d'un programme de taille relativement importante. Nous aurons aussi à produire un document écrit(rapport de TER), ainsi qu'un présentation orale, Soutenance publique en amphithéâtre.
\section{Déroulement d'un projet}
\begin{itemize}
\item Documentation, compréhension du sujet
\item Analyse du problème, conception de prototype papier, spécifications, analyse de besoin, tests.
\item Rédaction d'un rapport intermédiaire.
\item Architecture, conception.
\item Gros œuvre de l'implémentation logiciel
\item rédaction du mémoire final
\item Soutenance
\end{itemize} 
\section{Cours et TD} on aura des TD de communications, ainsi que des cours magistraux tels que Génie logiciel...\\\\
\section{Contenu du mémoire intermédiaire}
\begin{itemize}
\item résumé du projet
\item introduction du domaine.
\item Analyse de l'existant
\item Une liste et analyse des besoins non fonctionnels(+ risques, problèmes techniques, tests, essais).
\item Une liste et analyse des besoins fonctionnels (+prototype papier , problèmes techniques, tests, essais).
\item une description des prototypes et des résultats des tests préparatoires.
\item un planning, affectations des taches.
\item une bibliographie. c'est simplement la liste des références
\end{itemize}
\section{Besoins non fonctionnels}Qualités globales que l'on attend du logiciel.\\\\Par exemple ; domaines d’action, temps de réponse, fiabilité, facilité d'utilisation, convivialité, esthétiques des interfaces, langage, portabilité.\\\\Argumentés avec leurs risques et parades, une description des tests de validations, Un description du problème technique associé.
\section{Besoins fonctionnels} fonctionnalités nécessaires au logiciel.
\section{Mémoire intermédiaire} La description des prototypes, essais, et tests : 
\section{Rédaction des documents} Tous les documents doivent être rédigés en latex, car gestion des références bibliographiques facilités, permet de se concentrer sur le contenu
\section{Logiciel} Le programme doit etre propre, réutilisation de logiciels existants, Utilisation de logiciel de développement existants, Bon choix de structures de données, d'algorithmes, cloisonnement : abstraction modulaire, masquage d’implémentation , séparation interface/implémentation, composant réutilisable. Lisibilité , commentaire, pas de duplications, Robustesse, traitement des erreurs.\\\\
\section{contenu final} Ce qui 'il avais dans le mémoire intermédiaire, plus : 
\begin{itemize}
\item des exemples de fonctionenemetnt
\item l'architecture, le découpage modulaire,
\item Un descritpion, commentaire techniques, justification des algorithmes et les structures de données utilisées, 
\item evolution possible
\item pas de conclusion 
\end{itemize}  
\section{Contenu de la soutenance}
\begin{itemize}
\item Description du domaine
\item description générale du logiciel
\item architecture du logiciel
\item description de quelques point techniques
\item problèmes techniques "intéressants"
\end{itemize}
\section{le Client}
Notons que 65 \% de la note est sur les deux mémoires\\Mémoire intermédiaire  + maquette : 4 mars\\ mémoire final 27 mai 
\end{document}