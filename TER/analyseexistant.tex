\section{Analyse de l'existant}
\paragraph{}
R est un logiciel de statistiques open source qui fonctionne par modules
http://www.r-project.org/.
\paragraph{}
La communauté R est très active et de nombreux modules sont régulièrement proposés.
R permet facilement de charger des données, de sortir des informations statistiques, de visualiser
diverses courbes extraites des données, etc.
\paragraph{}
Par défaut, R se présente sous la forme d'une ligne de commande. Il est cependant possible de
développer des interfaces utilisateur pour contrôler l'application ainsi que pour visualiser des
résultats de traitement.
\paragraph{}
De notre côté, nous développons depuis plusieurs années un logiciel de visualisation de données
doté d'une interface graphique 3D interactive. Ce logiciel est écrit en Java.
http://www.univ-orleans.fr/lifo/software/Explorer3D/
\paragraph{}
Nous avons acquis une bonne maîtrise de la structuration d'un tel outil afin de le rendre évolutif.
Nous avons également mis en place un certain nombre de fonctionnalités interactives.
Nous étudions maintenant les possibilités d'interactions entre ce logiciel et la plateforme R, afin
d'intégrer rapidement divers outils existant sous R, mais également de diffuser notre logiciel vers
cette communauté.
\paragraph{}
Au moins deux approches sont possibles : 1/ intégrer des invocations de R depuis java. 2/ migrer
notre logiciel vers R.
\paragraph{}
Concernant le premier point, l'invocation de R depuis java est relativement simple. Elle ne sera pas
abordée dans ce TER. Concernant le second point, R propose la création de fenêtres de contrôle et
de fenêtres graphiques 2D et 3D (voir par exemple http://rgl.neoscientists.org/). Toutefois, nous ne
connaissons pas les possibilités d'interaction réelles de l'interface 3D.
\paragraph{}
L'objectif de ce TER consiste à :

• étudier le mécanisme de développement d'interfaces utilisateur sur le plan 1/ des fenêtres de
contrôle, et 2/ des fenêtres de visualisation 3D et 2D ;
• en produire une synthèse ;
• réaliser une maquette dont le contour précis sera discuté avec l'enseignant (charger un jeu de
données, visualiser en 3D après préparation des données, sélectionner des objets de la scène
3D grâce à la souris, afficher des images).

\newpage