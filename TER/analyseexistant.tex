\section{Analyse de l'existant}
\subsection{le logiciel R}
\paragraph{}
R est un logiciel de statistiques open source qui fonctionne par modules
http://www.r-project.org/.
\paragraph{}
La communauté R est très active et de nombreux modules sont régulièrement proposés.
R permet facilement de charger des données, de sortir des informations statistiques, de visualiser
diverses courbes extraites des données, etc.
\paragraph{}
Par défaut, R se présente sous la forme d'une ligne de commande. Il est cependant possible de
développer des interfaces utilisateur pour contrôler l'application ainsi que pour visualiser des
résultats de traitement.


\subsection{Le logiciel Explorer3D}
\paragraph{}
De notre côté, nous développons depuis plusieurs années un logiciel de visualisation de données
doté d'une interface graphique 3D interactive. Ce logiciel est écrit en Java.
http://www.univ-orleans.fr/lifo/software/Explorer3D/
\paragraph{}
Nous avons acquis une bonne maîtrise de la structuration d'un tel outil afin de le rendre évolutif.
Nous avons également mis en place un certain nombre de fonctionnalités interactives.
Nous étudions maintenant les possibilités d'interactions entre ce logiciel et la plateforme R, afin d'intégrer rapidement divers outils existant sous R, mais également de diffuser notre logiciel vers cette communauté.\\Au moins deux approches sont possibles :
\begin{itemize}

\item intégrer des invocations de R depuis java.
\item migrer notre logiciel vers R.
\end{itemize}

\paragraph{}
Concernant le premier point, l'invocation de R depuis java est relativement simple. Elle ne sera pas
abordée dans ce TER. Concernant le second point, R propose la création de fenêtres de contrôle et
de fenêtres graphiques 2D et 3D (voir par exemple http://rgl.neoscientists.org/). Toutefois, nous ne
connaissons pas les possibilités d'interaction réelles de l'interface 3D.
%\input{introex3D}

\newpage
\subsection{Packages R utilisables}
\subsubsection{RGL}
\paragraph{}
RGL est un package pour le langage de programmation R. Il étend les possibilités de R avec l'ajout d'outils de visualisation 3D en temps réel.
\paragraph{}
Les univers 3D ont besoin d’être projetées sur des images 2D affichées à l’écran, c'est pourquoi il est nécessaire d'avoir un outils qui effectue cette étape afin d'afficher des objets tri-dimensionnel à l’écran. Cet outils doit simuler la lumière, la structure des objets et leurs textures afin de donner l'illusion de 3D sur les images 2D projetées.
\begin{center}
\includegraphics[scale=0.7]{screen_rgl2.png}\\
\textit{Des sphères dans RGL}
\end{center}


\paragraph{}
Le coeur de RGL est codé en c++ et utilise OpenGL. RGL est ainsi une interface entre R et OpenGL. OpenGL est un standard des applications graphiques utilisé dans un grand nombre d'applications et langages. Il résout notamment le problème d'affichage en 2D d'univers 3D cité plus tôt. 
\paragraph{}
Plusieurs fonctionnalités sont accessibles simultanément via RGL, telles que l’intégration drag/drop dans la fenêtre de visualisation et la réception de nouvelles instructions depuis la console de commande de R. Ainsi les objets peuvent être examinés en trois-dimension grâce au zoom et la scène peut pivoter sur elle-même.

\begin{center}
\includegraphics[scale=0.7]{screen_rgl3.png}\\
\textit{Plus de sphères dans RGL}
\end{center}


\paragraph{}
Le but de RGL est donc de surcoucher OpenGL pour le langage R. Il permet ainsi via des instruction en R de générer des scènes 3D et d'interagir avec celles-ci via de nombreux outils.

\subsubsection{RGTK}
\paragraph{}
Le package RGtk2 s'ajoute en tant que librairie externe pour inclure dans R la possibilité de créer des interfaces graphiques. 

\paragraph{}
Comme dans Explorer3d, on a besoin de gérer les données grâce à l'interface graphique. Cette extension de R est essentielle dans le développement de l'application. 

 \begin{center}
\includegraphics[scale=0.4]{multipleviews.png}\\
\textit{fenêtre divisé en plusieurs parties}
\end{center}

\paragraph{}
RGtk est un package dérivé de GTK2. Il est écrit en c++. GTK est un standard pour la création d'interface utilisateurs en c++ et bien d'autres langages. Grâce à RGtk2 on résout le problème de création d'interface graphique. Comme dans la capture d'écran précédente on peut créer une fenêtre avec plusieurs compartiment à l'intérieur de celle ci. ON peut aussi avoir une fenêtre pour choisir la couleur d'un objet objet affiché dans une vue RGl. 

\begin{center}
\includegraphics[scale=0.4]{colorselector2.png}\\
\textit{Sélectionneur de couleur RGtk}
\end{center}

\paragraph{}
Le but de RGtk2 est de surcoucher GTK. Avec ce package on pourra interagir avec le logiciel. \\\\\indent

Nous trouvons ce package intéressant, car il nous offre la possibilité de créer des interfaces graphiques pour l'utilisateur, ce qui lui permettra de faire les demandes de calculs, sans passer par la console. 

\begin{center}
\includegraphics[scale=0.4]{regression.jpg}\\
\textit{Un autre exemple d'interface graphique avec RGtk}
\end{center}
\newpage