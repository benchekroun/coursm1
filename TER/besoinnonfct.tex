\section{Besoins non fonctionnels}
\begin{itemize}
\item Montée en charge
\item Interaction
\item Documentation des technologies à utiliser
\end{itemize}
\paragraph{Montée en charge} Comme expliqué plus haut, nous allons différer de la méthode utilisée dans Explorer3D pour toute la partie affichage. Pour savoir si les solutions choisies sont acceptables, il sera nécessaire de faire des tests de montée en charge.\paragraph{} Pour cela nous allons tester la réactivité du système aux différents événements, que ce soit des demandes de calculs,ou des interactions avec la semaine 3D (rotation, zoom, déplacement, etc...).\paragraph{} Pour réaliser ces tests , nous procéderons à la main aux différentes manipulation, et nous nous baserons a notre jugement en terme de vitesse de réponse, car il n'est pas facile de chronométrer la réactivité de la scène 3D, nous allons bien entendu refaire ces test avec un nombre de plus en plus grand d'objet à afficher, pour pouvoir déterminer une limite en terme de quantité d'objet représentable.
\newpage