\section{Besoins non fonctionnels}
\begin{itemize}
\item Montée en charge
\item Documentation des technologies à utiliser
\end{itemize}
\subsection{Montée en charge}
\paragraph{} Comme expliqué plus haut, nous allons différer de la méthode utilisée dans Explorer3D pour toute la partie affichage. Pour savoir si les solutions choisies sont acceptables, il sera nécessaire de faire des tests de montée en charge.

\paragraph{} Pour cela nous allons tester la réactivité du système aux différents événements, que ce soit des demandes de calculs,ou des interactions avec la semaine 3D (rotation, zoom, déplacement, etc...).

\paragraph{} Pour réaliser ces tests , nous procéderons à la main aux différentes manipulation, et nous nous baserons a notre jugement en terme de vitesse de réponse, car il n'est pas facile de chronométrer la réactivité de la scène 3D, nous allons bien entendu refaire ces test avec un nombre de plus en plus grand d'objet à afficher, pour pouvoir déterminer une limite en terme de quantité d'objet représentable.

\paragraph{}
L'objectif de ce TER consiste à :étudier le mécanisme de développement d'interfaces utilisateur sur le plan 1/ des fenêtres de
contrôle, et 2/ des fenêtres de visualisation 3D et 2D ;
• en produire une synthèse ;
• réaliser une maquette dont le contour précis sera discuté avec l'enseignant (charger un jeu de
données, visualiser en 3D après préparation des données, sélectionner des objets de la scène
3D grâce à la souris, afficher des images).

\subsection{Documentation des technologies}
\paragraph{} Dans la cadre cette étude nous allons être amener à utiliser certaines technologies qui ne seront pas forcément facile à mettre en oeuvre. Cette étude devant se placer dans la continuité d'un projet plus large, nous allons aussi devoir produire et introduire dans notre rendu, une documentation sur les technologies.\\\\\indent
Mais plus qu'une documentation technique, nous souhaitons produire plusieurs petits tutoriaux, qui permettrons à nos successeurs de gagner du temps, et de prendre en main plus facilement le résultat de notre travail, ainsi que les technologies utilisées pour mener à bien ce projet.
\newpage