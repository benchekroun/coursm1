\documentclass{article}
\usepackage[utf8]{inputenc} % un package
\usepackage[T1]{fontenc}      % un second package
\usepackage[francais]{babel}  % un troisième package


\title{ED- Christel Vrain}
\author{Alexandre Masson}
\date{14 Janvier 2013}

\begin{document}
\maketitle
\newpage
\tableofcontents
\newpage
\paragraph{} les TDs sont en anglais.
\section{Motivation}
\paragraph{}Plein de données, expansion de l'univers digital. différentes étapes : 
\begin{itemize}
\item statistique descriptive
\item Régression : analyser la relation d'une var vec d'autres
\item Fouille de données : Classification supervisée ou non, prédiction , recherche de motifs fréquents.\\\\
\end{itemize}
\textbf{univ lyon enseignements R}
\newpage
\section{Langage R}
\paragraph{} Langage et environnement pour le calcul statistique et les graphiques.\\ il tes libre.

\paragraph{commande de bases}
\textbf{Affectation} -> u <- ou assign("x",valeur).\\\\Des Variables : pas de chiffre ou caractère spécial en premier dans le nom, sensible à la casse, certains noms sont déjà pris.\\\\\textbf{structure de données : vecteur numérique} la structure élémentaire est i, vecteur, un nombre est un vecteur, on affecte a une variable un vecteur, avec l'opérateur c(...) en mettant dans c, le contenu du vecteur. l'opérateur [] donne l'élément du tableau , on peux passer  à [] un vecteur de données, il renvoi les valeurs placées à tous les index présents dans le tableau entre [].\\\\\textbf{Arithmétique vectorielle} toutes les opérations telles que sin, et cos, sont appliquées point à point sur tous les variables d'un vecteur. les tableaux sont indexés de 1 à N. la fonction sort(vecteur) est implémentée, ainsi que min et max, et range, length, sum, prod, mean.\\\\\textbf{NaN}  signifie not a number.
\paragraph{Vecteur Logique} les valeurs sont TRUE,FALSE,NA(not available). Nous avons a disposition des opérateurs logiques tels que <, <=, >, >=, ==, !=. nous avons aussi c1\&c2, c1|c2 et !c1. is.na() teste si X est NA ou NAN, is.nan(x) teste uniquement le nan.
\paragraph{Vecteur de caractères} les valeurs d'un vecteur doivent etre de meme type : logical , numéric, complex, character (ou raw). la fonction mode(x) renvoie le type des valeurs présentent dans x. on utlise as.character(x) pour transformer les valeurs de x en characters.
\paragraph{Index des vecteurs} un vecteur, l'opérateur x:y construit un vecteur remplis avec toutes les valeurs entre x et y. l’opérateur  - sur un vecteur renvoie le complémentaire du vecteur. il est possible avec names(vecteur) <- c("chaine", "chaine2",...] pour nommer les index des tableaux.
\paragraph{Séquences} fonction seq avec au plus 5 arguments : 
\begin{itemize}
\item from =
\item to = 
\item by = 
\item length = 
\item along = vecteur(seul argument si utilisé, créer une séquence de  1 à length(vecteur).
\end{itemize}

\paragraph{Factor} Utilisé pour étiqueter les données de vecteurs de même longueur., il est ensuite possible d'effectuer des opération en discriminant les valeurs par les étiquettes., tapply permet d'appliquer une fonction (3em param) sur un vecteur(1er param), en utilisant les étiquettes un vecteur d'étiquettes (2nd param).
\paragraph{Fonctions} définies de la forme nom <- function(arg1,...,argn). les instructions sont séparées par des ;. On a aussi accès aux conditionnelles, et au boucle, mais il faut éviter les boucles, les boucles for, repeat, while(condition) expr, et le break pour terminer les boucles.
\paragraph{Tableau - Matrices} Un vecteur peut être utilisé comme un tableau à plusieurs dimensions si on lui associe un vecteur de dimension. La fonction dim(vect) <- c(x,y); vect, transforme le vecteur en tableau de lignes et de y colonnes, il remplie ensuite le tableau avec les données du vecteur, en remplissant la première colonne avec autant de valeurs que de lignes, puis la seconde colonne avec la suite, etc...
\paragraph{Tableaux d'indices} il peux être construit par la fonction array(vecteur, dimension). on a aussi des opérations sur les matrices, soient a et b deux matrice et f une fonction n alors outer applique la fonction et renvoi une matrice de taille la concaténation des taille des deux vecteurs.
\paragraph{Listes} Collection ordonnée d'objets , appelés composant. Un composant peut être désigne par : \begin{itemize}
\item son numéro : Liste [[x]]
\item un nom : List \$name
\item ATTENTION : List[i:J] retourne une sous liste (avec les noms des composants).
\end{itemize}
Les listes sont extensibles, il est possible de rajouter des champs en mettant : List\$nomChamp<-valeur.
\paragraph{Data frame} c'est comme une matrice lais avec des modes et attributs différents, les chaines de caractères sont transformées en facteurs.
\paragraph{Lecture des fichiers} première ligne doit avoir une chaine de caractère par attribut du dataframe, tout les reste est ensuite lu comme attribut, mais les chaines sont considérées comme facteur.
\newpage
\paragraph{Réseau de neurones} Introduit dans les 60's. Les observations sont décrites par n variables et une étiquette 1-,1 ou 0,1. Les entrées sont reliées au neurones avec des poids, et l'étiquette c'est le produit scalaire des entrées et des poids, et si elle est plus grande qu'un poids on retourne un sinon zéro. Le neurone sépare donc l'espace en deux demi plan, un ou c'est vrai et un ou c'est faux. on note O pour output. Le perceptron est donc un ensemble de neurones qui sont connectés et où les sorties de neurones sont les entrées d'autres neurones.\\\\Apprendre le perceptron c'est apprendre les poids, on constate que les poids et le seuil ne sont pas du même coté de l'équation, on remplace le b par une entrée x0 toujours égale à 1, et un poids w0 qui devra être égale à b.\\\\\textbf{apprentissage par correction d'erreurs} \begin{verbatim}
Entrées : un ensemble d'exemples S de R^{n}*0.1
Sortie : P un perceptron défini par (w0,w1,...,wn)
Initialiser aléatoirement les poids wi
Repeter{
	Pour tout exemple (x1,x2,...,xn) dans S{
		calculer la sortie o
		pour tout i, wi <- wi + (c-o)xi
	}
jusqu'a convergence }
\end{verbatim}
Première question : converge-t-il? il a été montré que si les exemples sont linéairement séparable, il existe donc un hyperplan, il va le trouver. Dans la pratique, on va limiter le nombre de répétition des tours de boucle.\\ 2 critiques pour cet algorithme : \begin{itemize}
\item : algorithme ne converge que si les données sont linéairement séparables. Donc il converge s'il doit converger.
\item : pas de garantie sur la droite que l'on trouve, pas de distance maximum.
\item Cet algorithme s'appelle algorithme par descente de gradient. 
\end{itemize}

\section{Compléments sur R}
\subsection{Un peu de terminologie}
\begin{itemize}
\item Population: ensemble d'éléments sur lesquels se porte l'analyse
\item individu : élément de la population
\item variable : sert à décrire la population
\item modalité d'une variable : valeur de la variable
\end{itemize}
\paragraph{plot}
 plot(x,y), ou plot(u) avec liste de vecteur, ou matrice 2D.
\paragraph{variables qualitatives} 
Soucis de commit, penser à regarder le pdf graphique sous R.

\section{Graphique sous R}
\begin{center}
\date{6 Février 2013}
\end{center}
\paragraph{Continuons la partie représentation}
summary nous donne des infos sur une dataframe telles que le min, le 1st.quarter, la median, la mean, le 3rd. Quarter ainsi que le max. Faire attention, car le choix de découpage influe beaucoup sur la perception, et peut donner une fausse interprétation. faire plusieurs diagramme pour voir les données sous différents angles.\\Regarder d'un œil critique
\begin{itemize}
\item Mesures de tendances centrales ; moyenne, médiane, etc...
\item 
\end{itemize}
\paragraph{Peut être un cours de rappel de proba}
pas des proba pour faire des probas, mais par exemple revoir les lois gaussienne car on utilise des échantillons gaussiens 
\paragraph{Propriété de la moyenne}
Toutes les observations ont le même poids : les valeurs extrêmes influencent autant que les autres.\\Aussi la somme des écarts à la moyenne est nulle $\Sigma(x_{i} - mean(x)) = 0$ .\\ La somme des carrés des distances à la moyenne est plus faible que la somme des carrés des distances à toute autre valeur.\\-> la moyenne est la mesure qui minimise la somme des carrés des écarts à elle-même.\\\\\textbf{autres moyennes}\begin{itemize}
\item moyenne géométrique : racine $n_{ieme}$ du produit des n facteurs : x1*x2*...*xn
\item moyenne géométrique pondérée racine $k_{ieme}$ des $x_{n}^{k}$
\item moyenne harmonique : k éléments, divisé par la somme des 1/$x^{k}$
\item moyenne quadratique : Q= $\sqrt{1/k(x_{1}^{2}+x_{2}^{2}+...+x_{n}^{2})}$
\end{itemize}
Toutes ces moyennes se généralisent en : le nombre M tel que \\$f(M) = 1/n*(n_{1}.f(x_{1}+...+n_{n}.f(x_{n}))$
\paragraph{Médiane} 
point qui partage la distribution d'une série d'observations en deux parts égales -> partage l'histogramme(densité) en deux parties de surface égales.\\ Ne s'applique que pour des observations pouvant être ordonnées.\\ Minimise la somme des distances absolues de toutes les observations à ce point. \\\\\textbf{Estimation} Classement des observations par ordre croissant\\\textbf{Calcul : }si le nombre d'observations est impair, la médiane est la (k+1)/2 valeur\\Si c'est paire, toute observation entre k/2 et(k/2)+1, ou la moyenne des observations k/2 et (k/2)+1.\\ si les observations sont regroupées par classe, une valeur plus précise sous l'hypothèse d'une répartition uniforme des observations.

\paragraph{Mode}
Valeur qui possède la fréquence la plus élevée\\n'est pas toujours une valeur centrale de la distribution\\une distribution peut avoir plusieurs modes, unimodable, bimodale, asymétrique...
\paragraph{Comparaison}
\begin{itemize}
\item moyenne : tiens compte des valeurs de la distribution
\item mode : indique une seule valeur de la distribution
\item médiane : indique un rang
\item si la distribution est unimodale et symétrie, moyenne , mode, et médiane sont confondus.
\item Si la distribution est bimodale et symétrique, moyenne et médiane sont confondues.
\item si la distribution est asymétrique, ils peuvent avoir des valeurs différentes.
\begin{itemize}
\item tirée a droit, mode, puis médiane , puis moyenne
\item tirée a gauche : moyenne , médiane, mode
\end{itemize}
\end{itemize}
\paragraph{Variance}
Elle donne la répartition des valeurs autour de la moyenne, si elle est faible, l'ensemble des valeurs est proche de la moyenne, sinon, cela nous indique que la moyenne c'est pas représentative des valeurs des observations.\\\textbf{Propriété}\\la variance est toujours positive\\ si elle est nulle , toutes les observations sont identiques,\\si on ajoute la même constante a toutes les observations, alors la moyenne change, mais pas la variance.\\si on multiplie les observations par une même constante positive ou négative, on modifie la variance à un facteur multiplicatif(le carré de la constante, vu que la variance fait un carré)
\paragraph{Autre façons de calculer la variance} variable quantitative avec k observations, x1,...,xn = 1/k*$\Sigma (x_{i}^{2}-mean(x)^{2})$
Différence entre var de R ,et la vrai variance, car R se base sur un échantillon pour trouver la moyenne pour effecteur son calcul. On parle d'estimateur sans biais, si quand on change d'échantillon, on retrouve la même valeur qu'avec toue la population, donc la variance ne l'est pas.
\paragraph{Autres mesures}
écart moyen : prend la valeur absolue plutôt que le carré\\
écart médian : prend la médiane au lieu de la moyenne pour le calcul de variance.
intervalle interquartile : \begin{itemize}
\item Intervalle comprenant 50 \% des observations le plus au ventre de la distribution.
\item Se définit à partir des quantiles qui sont des positions particulières (ex :médiane)
\item quartile : coupe en quatre, déciles:  amis en 10, centiles : idem mais en 100.
\item IQR : intervalle inter quantile : il fait la différence entre le 75\% et le 25\%
\end{itemize}
\textbf{Dispersion pour les variables qualitatives}
\begin{itemize}
\item variables dichotomique : même définitions que pour les variables qualitatives
\item variable multicatégorielles : $\sigma^{2}(X) = p_{1}*...*p_{j}$ ou $p_{i}$ est la proportion d'apparition de la classe.
\end{itemize}


\paragraph{Coefficient de Pearson}
\date{13 février 2013}
\paragraph{Moment d'ordre r, en langage R} $sum[(x-mean(x))^{3})/length(x)]^2$ 
\paragraph{mesures d'aplatissement}
coefficient de Pearson $\beta_{2}$ = $\frac{u^{4}}{u_{2}^{2}}$ d’autant plus faible que la courbe est platicurtique.\\coefficient de Fisher $\gamma{2}=\beta_{2}-3$

\section{Rappels de probabilité}
\subsection{Notion de base}
\paragraph{} 
Expérience aléatoire : on connaît l'ensemble des résultats possible, on ne peut prédire le résultat\\ Ensemble fondamental $\Omega$ de tout les résultats possibles de l'expérience\\Événement : résultat , événement vide = $\emptyset$
\paragraph{exemple : lancer de deux dés successifs} 
\begin{itemize}
\item événement vide $\emptyset$  :somme des deux lancer =  0
\item événement certain $\Omega$ somme < 13
\end{itemize}
\paragraph{Opérations} 
\begin{itemize}
\item Négation
\item conjonction
\item disjonction 
\end{itemize}

\textbf{Propriétés: Axiome}
\begin{itemize}
\item  $p(\Omega) = 1$.
\item $p(\emptyset)=0$
\item $p(!A)=1-p(A)$
\item $p(A \cup B) = p(A)+p(B)$
\end{itemize}

\textbf{Exemple : deux billets pour quatre enfants, 2 filles(b et d), deux garçons (a et c),  quelle probabilité de choisir un gars et une fille?}
\begin{itemize}
\item événement $\Omega$ : on choisit deux personnes {a b, a c, a d, b c, b d, c d}. Donc chaque couple a une probabilité de $\frac{1}{6}$
\item E = {a b, a d, b c, b d}. donc $\frac{4}{6}$
\end{itemize}
\paragraph{probabilités conditionnelles}
probabilité de B sachant A \\ 
$P(B|A) = \frac{P(A \cap B)}{P(A)}$\\
\\
\textbf{Théorème de Bayes} $P(B|A).P(A) = P(A|B).P(B)$
\paragraph{Exemple concret} classification supervisée\\
Utilisation du classifier bayésien.\\
Il nous dit que les données sont décrites par des attributs $A_1,A_2,...,A_n$.\\Et sont étiquetées par leur classe.\\
Étant donnée une observation $A_1=v_1, A_2=v_2,...,A_n=v_n$\\
déterminer la classe.
\\
\begin{center}
$argmax_{C \in Classes} P(C|A_1=v_1, A_2=v_2,...,A_n=v_n$) =\\
$ argmax_{C \in Classes} \frac{P(A_1=v_1, A_2=v_2,...,A_n=v_n|C)*P_{r}(C)}{P(A_1=v_1, A_2=v_2,...,A_n)}$\\
=$ argmax_{C \in Classes} P(A_1=v_1, A_2=v_2,...,A_n=v_n|C)*P_{r}(C)$\\
\end{center} 
2 classes, 2 attributs $A_1,A_2$ qui ont 2 valeurs\\ 
$2+ 4*2 = 10$ probabilités à calculer, c'est beaucoup\\
\textbf{Hypothèse : }attributs indépendants/à la classe\\
$P(A_1=v_1, A_2=v_2,...,A_n=v_n|C)=P(A_1=v_1|C)*...*P(A_n=v_n|C)$
\paragraph{Indépendance} $P(A \cap B) = P(A)*P(B)$
\paragraph{Indépendance/à la classe} $P(A \cap B | C) = P(A|C) * P(B|C)$
\subsection{\date{20 Février 2013}}
\paragraph{}On rappelle que l'ensemble $\Omega$ est l'ensemble des événements.\\\\\indent\textbf{fonction de répartition} $F(x) = P(X\leq x)$
\begin{itemize}
\item F est croissante
\item $\forall x , F(x) \in [0,1]$
\end{itemize}
\paragraph{Formule}
Espérance  : $E(X)= \sum x_i p(x_i)$\\
Variance $\sigma^2 = Var(X) = \sum (x_i -\mu)^2 . p(x_i) = E(X-\mu)^2$\\\\\textbf{Propriété de l'espérance et de la variance}
\begin{itemize}
\item E(a.X + B) = a.E(X) +b
\item si X et Y sont indépendants : E(X.Y) = E(X).E(Y)
\item Var(a.X+b) = $a^2$.Var(X)
\end{itemize}
\indent \textbf{loi conjointe , loi marginale}
X : nombre de piles, Y : nombre de pile dans les deux premiers essais. On considère donc les probabilités d'avoir les événements X et Y, à savoir quel(s) événement(s) satisfont Y et X. si x et y sont indépendants $ p(x,y)  = p(x) * p(y)$. la covariance de deux ev indep est  0 , donc on calcule souvent cette donnée sur nos observations pour savoir si des ev sont indépendants.\\E(X) =$0*\frac{1}{8}+1*\frac{3}{8}+2*\frac{3}{8}+3*\frac{1}{8} = 12/8 = 3/2$\\\\E(Y) = $0*\frac{2}{8}+1*\frac{4}{8}+2*\frac{2}{8}= \frac{8}{8} = 1$\\\\
\\loi binomiale : somme d'une série de Bernoulli (ensemble d'ev suivant la même loi)\\Probabilité d'avoir x succès sur n tirage : $P(X=x)=C_n^x * p^x * q^{n-x}$ avec $C_n^x = \frac{n!}{x!(n-x)!}$\\$\mu = n * p$ (somme de n variables de Bernoulli indépendantes)\\$\sigma^2 = n*p*q$\\\textbf{Exemple avec R} \\ 
rXXXX:  génération de nombres aléatoires suivant la loi XXXX\\dXXXX étant donné un ensemble de valeurs, retourne la hauteur de la probabilité de distribution à chaque point.\\pXXXX : probabilité P(x$\leq$x) qu'un nombre tiré aléatoirement suivant cette loi soit inférieur à un nombre x donné.\\qXXXX : nombre X tel que P(X$\leq$x) soit égale à n , qui est donné.
\subsection{Variable aléatoire continue} 
Machine avec 24 composants \\
probabilité qu'un composant tombe en panne : 0.2\\
la machine fonctionne quand au moins deux tiers des composants sont en marche.\\
Probabilité que la machine fonctionne.\\X nombre de composants en marche.\\P(X$\geq$16)=$\sum_{i=16}^{24} P(X=i) = \sum_{i=16}^{24} C_{24}^i (0.8)^i (0.2)^{24-i}$
\end{document} 