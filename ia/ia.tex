\documentclass{article}
\usepackage[utf8]{inputenc} % un package
\usepackage[T1]{fontenc}      % un second package
\usepackage[francais]{babel}  % un troisième package


\title{IA- Matthieu Exbrayat}
\author{Alexandre Masson}
\date{14 Janvier 2013}

\begin{document}
\maketitle
\newpage
\tableofcontents
\newpage
\section{Organisation du cours}
\paragraph{} Cours de 2h le mardi et TD 2 h le mercredi.\\\\
\section{Introduction}
\paragraph{Questions} Qu'est ce que les taches suivantes ont en commun ?\\\\
Concevoir des systèmes capables de faire des choses compliquées.\\Il y a aussi l'aspect ; apprentissage , le système est il capable d'apprendre de lui même et se débrouille tout seul.\\Exemples:\\Concevoir un programme qui vire tout seul les spam dans les mails.\\Concevoir un super navigateur qui s'occupe tout seul de faire les mise à jour logicielle.\\\\Le point commun? Posséder un certain degré d'intelligence.\\D'où grande question qu'est ce que l'intelligence ? 
\begin{itemize}
\item Selon Darwin : Ce qui permet l'individu le plus apte, parfaitement adapté a son environnement
\item Selon Edison : ce qui fonctionne et qui produit de l'argent. 
\item Selon Turing : ce qui rend difficile la distinction entre une tache réalisée par un être humain ou une machine
\end{itemize}
L'IA est une discipline qui systématise et automatise les taches .
\begin{itemize}
\item Penser comme un humain
\item Agir comme un humain
\item Penser rationnellement
\item Agir rationnellement
\end{itemize}
\paragraph{Penser comme un humain} Il faut comprendre l'esprit Humain.\\Comparaison des différentes étapes d'un programme et du raisonnement humain pour arriver au même problème.\\Science cognitive.
\paragraph{Agir comme un humain} Test de Turing, on transforme l’ordinateur peut il penser en peut il se comporter intelligemment. Le test est concluent si l'opérateur ne sais pas si la réponse a ses question est donnée par un humain ou une machine
\paragraph{Penser rationnellement : approche logique}
\newpage
\paragraph{Agir rationnellement : agir pour atteindre un objectif}Remplir une mission de la meilleure façon.\\\\Agent : unité qui fonctionne de façon autonome, perçoit son environnement , s'adapte aux changements et est capable d'atteindre un objectif.\\\\Agent Rationnel : agent qui agit pour atteindre le meilleur résultats ou du moins le meilleur résultat espéré.
\paragraph{Loebner prize} Tous les ans récompense le meilleur système du test de Turing.
\begin{itemize}
\item traitement du langage naturel : pouvoir communiquer en un langage naturel.
\item représentation de connaissances : stocker ce qu'il sait ou ce qu'il perçoit.
\item raisonnement automatique : utiliser des connaissances pour répondre aux questions.
\end{itemize}

\paragraph{les Lois de la pensée}Aristote : quels sont les processus de pensée corrects ?\\Plusieurs formes de logique : notation et règles de dérivation pour les pesées , sans relations avec une mécanisation du raisonnement.\\Ligne directe entre math et philo -> IA.\\Problèmes : tous les comportements intelligents ne sont pas le résultats d'un raisonnement logique.\\
\paragraph{Agent Rationnel} abstrait : un agent est une fonction qui met en correspondance des séquences perceptives P* et des actions A : f : P* -> A.\\\\processus où on ne sais pas tout, ou temps limité trop court pour répondre, donc réponse acceptable mais pas optimale.
\paragraph{Fondements de l'IA}Elle repose sur plusieurs domaines.\\\\Philosophie : logique et raisonnement.\\\\Mathématiques : représentations formelles et preuves, algorithmes, (in)décidabilité, probabilité.\\\\Psychologie : adaptation, perception et controle moteur, techniques expérimentales.\\\\Linguistique : représentation de connaissances grammaires.\\\\Neurosciences : substrat physique et biologique de l'activité mentale.\\\\Théorie du controle : systèmes asservis, stabilité, concept d'agent optimal.

\section{Histoire de l'IA}\begin{itemize}
\item 1943 : modélisation de neurones.
\item 1950's : vision complète de l'IA
\item 60's : dev des réseaux de neurones.
\item 70's : développement des systèmes à base de connaissances, faire de l'aide au diagnostic.
\item fin 80's : hiver de l'IA, l'ia s’effondre
\item depuis 20 ans : IA évolue et révolutionne sa méthodologie.
\item plein de fric à se faire dans la Bourse, algorithme qui réagissent au plus vite et essaye d'analyser la bourse pour établir des règles. 
\end{itemize}

\paragraph{Prédictions et réalité} différence prédictions réalité ;,  60's œil électronique, pas encore fait mais ça avance et s'est encourageant.\\\\ Robot qui font tout, déjà dans les 70's.
\section{Qu'est ce qu'un problème pour l'IA}Problème qui n'as pas de solution analytique connue, objectif : si l'objectif est hors du possible donner une solution acceptable en temps raisonnable.\\\\Certaines taches aisées pour l'humain sont difficiles pour la machine : tout ce qui a besoin de la sensibilité humaine.

\paragraph{Contenu du cours} 4 approches différentes de la résolution de problèmes en IA
\begin{itemize}
\item recherche de solution dans un espace d'états
\item recherche par raisonnement et en présence d'incertitude
\item résolution par planification.
\item résolution par apprentissage automatique
\end{itemize}

\section{Agents Intelligents}
\paragraph{Qu'est ce qu'un agent}L'AGENT perçoit des choses de son ENVIRONNEMENT, à laide de SENSORS,  il accompli ensuite sa MISSION, et utilise ses ACCUATORS, pour effecteur des actions.\\\\Définition : entité capable de percevoir son environnement par des capteurs et d'agir sur son environnement à l'aide d’effecteurs (actionneurs). AUTONOMIE , PERCEPTION,ACTION,OBJECTIF, sont les maitres-mots qui définissent l'existence de l'agent.
\paragraph{spécification d'un agent} le choix d'une action à l'instant t dépend de séquences perceptives .
\paragraph{Agent rationnel}Basé sur le raisonnement.\\\\Toute option considérées , faire le meilleur choix pour maximiser les chances de succès.\\\\Mesures de performances : réussir la tâches, quantificatino maximale d'un objectif.
\paragraph{spécifier l'environnement} spécif = probleme, agent ) solution
\paragraph{PEAS} exemple , conduite automatique 
\begin{itemize}
\item P (mesure de performance) : bon endroit, temps, sécurité
\item E (environnement) : rues, voitures, piétons, météo
\item A (actions) : tourner , s'arreter , klaxonner, etc...
\item S (senseurs) : plein.
\end{itemize}
\end{document}