\documentclass{article}
\usepackage[utf8]{inputenc} % un package
\usepackage[T1]{fontenc}      % un second package
\usepackage[francais]{babel}  % un troisième package


\title{ TD IA- Matthieu Exbrayat}
\author{Alexandre Masson}
\date{14 Janvier 2013}

\begin{document}
\maketitle
\newpage
\tableofcontents
\newpage
\section{Feuille 1}
\subsection{Exercice 1}
\paragraph{Question 1 : suite d'actions} Aspirer-Droite-Aspirer-gauche-aspirer
\paragraph{Question 2 : montrer que la fonction est rationnelle} $\forall$ cases, if sale , aspirer, donc objectif nettoyer toutes les cases.\\oui car définition explicite de l'environnement\\oui, car aucun autre return\\oui car il reçoit position et prop.
\paragraph{Question 3 : écrire un agent qui fais la même chose mais chaque déplacement coûte un point} 
\begin{verbatim}
function Agent-reflex([position,prop]) returns action
	if position already in base de connaissance
		NoOp
	else
		if Sale return Aspirer
		else
			if position = A return droite
			else
				return gauche
\end{verbatim} 
Mais dans ce cas, il nécessite de garder une variable coût.
\paragraph{Question 4 : même question mais l'agent connais l'état de toutes les cases}
\begin{verbatim}
function Agent-reflex([position,ETAT]) returns action
	if ETAT[position] = Sale then return Aspirer
	else{
			if position = A then {
				if Etat[B] = Sale then return droite;
				else return NoOp;
			}
			else{
				if Etat[A] = Sale then return gauche;
				else return NoOp
		}
\end{verbatim}
\paragraph{Discuter des descriptions possibles dans le cas où une case propre peut redevenir sale et ou l'environnement géographique n'est pas connu} IL faut que l'agent mette a jour sa carte du monde au fur et a mesure, approche exploratoire
\paragraph{PEAS}
\begin{itemize}
\item P : Objectif : cases propres, capteurs données des perceptions correctes, pas de connaissance sur le coût des actions, les actions ont un vrai effet, 
\item E : deux cases, A et B, A à gauche de B, les cases sont communicantes, les cases peuvent être sales ou propres, environnement évolue de case propre->sale
\item A : quatre actions: Aspirer, Gauche, Droite, NoOp,
\item S : 2 capteurs propreté, position.
\end{itemize}

\section{Feuille 2}
\subsection{Exercice 1}


\section{4 Avril 2013}
\end{document}