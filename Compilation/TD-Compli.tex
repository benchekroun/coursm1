\documentclass{article}
\usepackage[utf8]{inputenc} % un package
\usepackage[T1]{fontenc}      % un second package
\usepackage[francais]{babel}  % un troisième package
\newcommand{\test}{Ceci est un test de macro avec LaTeX et un backslash : $\backslash$}

\title{TD - Compilation}
\author{Alexandre Masson}
\date{14 Janvier 2013}

\begin{document}
\maketitle
\newpage
\tableofcontents
\newpage
\section{Exercice 1 : Rappels de théorie des langages}
\subsection{Définitions}
\begin{itemize}

\item un alphabet : un ensemble de symboles atomique.
\item un langage : un ensemble fini (ou infini) de mots reconnu sur un alphabet donné
\item une grammaire : quadruplet avec , ensemble de symboles terminaux, ensemble de non terminaux, ensemble de règles, axiome de départ.

\end{itemize}

\subsection{la disparition} 
Soit $\Sigma$={a,...,d,f,...,z}, sire si le mot est dans $\Sigma$* dire si les mots suivants sont dedans : 
\begin{itemize}
\item Bonjour : non, il manque les majuscules.
\item $\epsilon$ : oui , $\epsilon$ (mot vide) $\in$ tout langage.
\item bonjour : ok.
\item message : il manque le e minuscule dans l'alphabet.
\item voici un message : il manque toujours le e minuscule, ainsi que le caractère espace.
\end{itemize}

\section{Exercice 2 : Compilation}
\subsection{rappelez les différentes phases d'analyse d'un compilateur}
\begin{itemize}
\item lexicale : scanner,
\item lexicale : cibler,
\item syntaxique : parser,
\item Analyse sémantique.
\end{itemize}
\subsection{spécification d'analyse lexicale}
Backus-Naur Form
\begin{itemize}
\item PHRASE ::= PHRASE PONCTUATION MOTS | MOTS | $\epsilon$
\item MOTS ::= CARACTERE MOTS | CARACTERE | $\epsilon$
\item PONCTUATION ::= . | , | ! | ?
\item CARACTERE ::= {A..Za..z} 
\end{itemize}

Lex
\begin{itemize}
\item mots [a-df-z]+
\item ponctuation ,|;|$\backslash$:
\item point .|$\backslash$!|$\backslash$?
\item endOfLine $\backslash$$\backslash$n
\end{itemize}
\test\\\\
\newpage
la fonction yylex() est celle qui va vérifier syntaxiquement le texte analysé, c'est pourquoi à la suite des instructions de reconnaissance de lexèmes, on ajoute des "règles" qui donnen explicitement a lex les instructions C a effectuer dans le cas ou il trouve un lexeme.\\\\On rajoute un [[:blank:]] et rien derriere pour lui faire ignorer les blank (espaces)
\end{document}