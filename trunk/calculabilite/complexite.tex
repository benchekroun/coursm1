\documentclass{article}
\usepackage[utf8]{inputenc} % un package
\usepackage[T1]{fontenc}      % un second package
\usepackage[francais]{babel}  % un troisième package
\usepackage{lmodern}
\usepackage{amsmath}
\usepackage{amssymb}
\usepackage{mathrsfs}


\title{ Complexité - Nicolas Ollinger}
\author{Alexandre Masson}
\date{15 Janvier 2013}

\begin{document}
\maketitle
\newpage
\tableofcontents
\newpage

\section{non-déterminisme}
\paragraph{} $P = \cup _{k>0} DTIME(n^k)$ la classe des problèmes qu'on peut résoudre en temps polynomial.\\
$NP = \cup _{k>0} NTIME(n^k)$ la classe des problèmes vérifiables en temps polynomial

\paragraph{Rétrospective} DFA : automate déterminisme , NFA non déterminisme.\\
ex :  $(a+b)^*baba$ mot finissant par baba. Tout NFA non déterminisme, se code en DFA, mais au prix d'une complexité plus importante en nombre d'état.
\paragraph{Définition} une MTND (MT non déterministe), est une MT avec comme table de transition : \\
\begin{center}
$T \subseteq (Q x \Gamma^k)^{input} x (Q x \Gamma^k x \{<-, v, ->\})^{output} $\\
\end{center}
on redéfinie la relation de transition : 
\begin{center}
$c \vdash c'$ si T contient une transition applicable sur c qui produit c'\\
\end{center}
Une exécution d'une MTND partant d'une entrée u est une suite de transition valides de la configuration initiale associée à u à une configuration d’arrêt.
\begin{center}
$c_0 \vdash c_1 \vdash ...  \vdash c_n$\\
\end{center}
Une MTND $M$ accepte un mot $u \in \Sigma^*$ s'il existe une exécution acceptée pour u.\\

\begin{tabular}{|c|c|c|}
\hline 
  & MT & MTND  \\ \hline
Rejet & l'exécution doit s’arrêter sur $q_{non}$ & toutes les exécutions doivent s’arrêter sur $q_{non}$  \\ \hline
Acceptation & l'exécution s’arrête sur $q_{oui}$ & un seul état à oui renvoie oui  \\ \hline
\end{tabular}

\paragraph{remarque} A un ralentissement linéaire près, on peut supposer que les choix non déterministes sont au plus binaires : pour tout configuration c: 
\begin{itemize}
\item ou bien c n'as pas d'image (arrêt)
\item ou bien  $c \vdash c'$ unique.
\item ou bien $ c \vdash c' \& c \vdash c''$
\end{itemize}

\paragraph{exemple } le 3-color, entrée : un graphe G=(V,E), sortie : G est il  3-colorable \\3Color est reconnu par une MTND dont les exécutions s’arrêtent en temps polynomial en la taille de l'instance.\\la machine fonctionne en deux temps.
\begin{itemize}
\item Choix non déterministe d'une couleur par sommet (O(|v|)
\item Vérification déterministe que le coloriage choisit est valide : si oui accepte, sinon rejette.
\end{itemize}

\paragraph{Définition} NTIME(f(n)) est la classe des langages reconnus par une MTND dont toutes les exécutions sont de longueur $\leq$ f(n) sur les entrées de taille n.

\paragraph{proposition} Si $f(n) \leq g(n)$ alors $NTIME(f(n)) \subseteq NTIME(g(n))$\\$DTIME(f(n)) \subseteq NTIME(f(n))$\\$NTIME(f(n)) \subseteq DTIME(2^{O(f(n))})$\\ \textbf{Idée} : Il suffit de simuler les choix de la MTND par du backtracking\\\\\textbf{def} $NP = \cup _{k>0} NTIME(n^k)$\\$NEXP = \cup _{k>0} NTIME(2^{n^k})$\\Question : Est ce que P NP, \textbf{rq :} $P\subseteq NP$\\\textbf{prop} NP ets clos par union et intersection.\\\textbf{def} $co\Im = \{!L | L \in \Im\}$ ou $\Im$ et une classe de langages.

\paragraph{} Un problème B est un \textbf{certificateur} pour un problème A si $\forall x \in \Sigma^{*}, x \in A <=> \exists y\in \Sigma^{*} <x,y> \in B$ on dit que y est un certificat pour x.

\paragraph{thm} Np la classe des langages qui possèdent un certificateur dans P avec certificats de longueur polynomiale en la longueur de l'instance. \\\textbf{ex } 3Color $\in$ NP , cf exemple antérieur. certificateur pour 3COLOR.\\{sur l'entrée <G,C> avec colorisation des sommets vérifier que les arêtes ont des couleurs distinctes à chaque extrémité}. Ce certificat est dans P( car il ne fais que parcourir les arêtes pour vérifier une propriété). la taille du certificat est polynomiale. 

\paragraph{Démonstration} 
\subparagraph{i} (dans ce sens, on connait A de P , et B le certificateur, on essaie de construire un chemin qui accepte x dnas la machine M, ce chemin s'appelle y , si l'on suis le chemin y pour arrive en qoui ave x en entrée, alros y est certificat et le certif existe , donc B est certificateur.)\\Soit A un langage qui possède un certificateur B $\in$ P avec certificat de longueur $\leq$ p(n) sur les instances de longueur n ou p est polynomiale en n.\\montrons que A ets dans NP en construisant une  MTND $M$ qui reconnait A comme suit : 
\begin{itemize}
\item choisir grace au non déterminisme un mot y de longueur $\leq$ p(|x|)
\item exécuter une MT qui reconnait B sur l'entrée <x,y> 
\item accepter si (2) accepte
\end{itemize}
$M$ s'exécute bien en temps polynomial en |x| et $M$ accepte x ssi $\exists y : |y| \leq p(n)$ et <x,y> $\in$ B, si x$\in$ A.

\subparagraph{ii} ( dans ce sens, on sais que le Y sera un certificat, puisqu'on chosiit un chemin acceptant de x dans l'exécution de M, donc si B repond ok , vu que le Y ets ok , cela veux dir que B est bon)\\Soit $A\in NP$ On va construire un certificat $B\in P$ pour A avec certificats polynomiaux.\\Soit $M$ un MTND qui reconnais A. \\On utilise les exécutions acceptantes de $M$ comme certificats et B se content de vérifier l'exécution. Sur l'entrée <x,y>B vérifie que Y est une exécution acceptable pour $M$ sur l'entrée x. B est bien un certificateur dans P avec certificat polynomiaux pour A.\\
Np est la classe des problèmes vérifiables en temps polynomial.

\subparagraph{} 3Color : E : graphe G, S : est il 3Colorable, certif : coloration, Verif : pas d'arete monochrome.\\\\Hamilton : E/S idem : certif  : une permutation des sommets, i.e le cycle, verif : les aretes existent bien.\\\\\textbf{CLIQUE(sous graphe complet du graphe)} Entrée:  un graphe et un entier k :  certif : la clique, verfi : les aretes existe =elles\\
\\
\\
\\\textbf{SAT (clause satisfiables)} entrée: formule $\Phi$ en CNF, sortie  : $\Phi$ est elle satisfaisable ? \\ variable : x,y,z, littéraux : var ou neg.var, clause : disjonction de littéraux, formules : conjonction de clauses, \textbf{certificat:} une affectation qui satisfait $\Phi$, vérif: évaluer $\Phi$ \textbf{3SAT} SAT où $\Phi$ ont des clause d'ua moins 3 littéraux.\\\\
\textbf{COMPOSITE} entrée : entier n en binaire, question : existe il p et q > 1 tels que n=pq?, certif : la paire <p,q> en binaire. vérif : multiplier p par q et comparer à n.

\section{19 Mars 2013}
\subsection{NP-Completude}
\paragraph{} Au menu : théorème de Cook-Levin SAT$\in$ P <=> P=NP\\remarque sens <= trivial car SAT $\in$ NP.\\\textbf{Définition} $A \leq_p B $ si $\exists f : \Sigma^* -> \Sigma^*$ totale calculable en temps polynomiale telle que :$\forall x \in \Sigma^*, X\in A <=> f(x) \in B$ \\\\\textbf{prop} si $A \leq_p B $ et B $\in$ P alors A $\in$ P. Que B soit dans P implique qu'il existe un algo poly qui résout B, \\\\\textbf{def} on dit que A est NP-difficile si quelquesoit le problème que je prend dans NP , ce problème est plus simple que A.( $\forall B \in NP, B \leq_p A$ \\A est NPComplet si \\A est NP-difficile \textbf{ET} A$\in$ NP.(\textbf{NE PAS L'OUBLIER})
\paragraph{} Remarque  : si A est NP-complet, et A $\in$ P, alors P=NP. En effet soit B $\in$ NP, comme A est NP-difficile, d’après la def, $B \leq_p A$ et d’après la proposition ci dessus, B$\in$P . Donc $NP \subseteq P$, i.e P=NP.
\paragraph{thm Cook-Levin} SAT est NP-complet. On sais que SAT $\in$ NP, on va montrer que $\forall A \in NP, A \leq_p SAT$\\soit $A \in NP$.soit $M$ une MTND totale qui reconnaît A en temps polynomial p(n). \\On peux supposer sans perte de généralité que M à  un seul ruban, à choix binaire, à ruban semi infini, avec arrêt sur la case 0, exactement au temps p(n). \\A l'instant 0 , le ruban contient le mot de taille N, si le temps d’arrêt est p(n), alors la ruban ne peux etre de taille plus grand que p(n) , car on parcours une case par étape , donc pour un mot donnée, on av obtenir des $q_{oui}$ ou des $q_{non}$ , $x \in A $ ssi $\exists$ un digne-espace temps de $M$ sur l'entrée X, avec $q_{oui}$ en (0,p(n)).
\paragraph{Idée:} Pour montrer que $A \leq_p SAT$ on construit f, fonction de réduction , qui a un mot $x\in \Sigma^*$ associe une formule CNF $\Phi_x^{M,p}$ qui code les diagrammes espace-temps de $M$ sur l'entrée x de sorte que $\Phi_x^{M,p} \in SAT$ ssi le diagramme possède des choix non-déterministes acceptant , c’est a dire avec $q_{oui}$ en (0,p(x)).\\\\On va découper $\Phi_x^{M,p}$ en plusieurs parties :\\
 $\Phi_{sim}^{M,p(|x|)} (calcul de la MT)\wedge \Phi_{input}^{M,P,x}(entrée) \wedge \Phi_{accept}^{M,P(|x|)}(sortie)$
  \\\\Pour chaque case du diagramme, elle sont codées par les variables booléennes, $x_1^{ij},...,x_k^{ij}$, en binaire.
  \\les codes valides sont vérifiés par une formule CNF $\Phi_{val}(x_1^{ij},...,x_k^{ij})$\\
  $\Phi_{sim}^{M,p(|x|)} = \wedge_{i=0}^N \wedge_{j=0}^N \Phi_{val} (x_1^{ij},...,x_k^{ij}) \wedge \Phi_{trans}^{M,N}$\\
  la variable $C_t$ code le choix non-det à l'instant t. (si le contenu est loin de la tête , il n'a aucune chance de changer entre t et t+1.\\
  $\Phi_{trans}^{M,N} = \wedge_{i=0}^N \wedge_{j=0}^N \Phi_{delta}^{M,N}(x_1^{i-1j},...,x_k^{i-1j},x_1^{ij},...,x_k^{ij},x_1^{i+1j},...,x_k^{i+1j},x_1^{i-1j+1},...,x_k^{i-1j+1},$\\$
  x_1^{ij+1},...,x_k^{ij+1},x_1^{i+1j+1},...,x_k^{i+1j+1})$
  
  où $\Phi_{delta}^M$ vérifie que que le rectangle 3x2 cellules communiqués est une transition valide pour M\\\\On obtient une formule CNF $\Phi_{sim}^{M,N}$ à $kN^2$ variables et de taille O(N$^2$). On est tous convaincu de savoir produire $\Phi_{sim}^{M,N}$ en temps poly.\\\\
$\Phi_{input}^{M,P,x}$ vérifie que la ligne 0 code via la config initial\\
$\Phi_{input}^{M,P,x} = \Phi_{case}^M(q_0,x_0),x_1^{00},...,x_1^{00} \wedge \wedge_{i=1}^{|x|-1} \Phi_{case}^M(x_i,x_1^{i0},...,x_k^{i0}) \wedge \wedge_{i=|x|}^{p(|x|)} (B,x_1^{i0},...,x_k^{i0})$ formule de ligne O(n)$^i=|x|$ est calculable en temps polynomial\\\\

$\Phi_{accept}^{M,N}$ vérifie que la case (0,n] contient bien $q_{oui}$ \\
la formule $\Phi_x^{M,N}$ est donc calculable en temps polynomial en |x| et est satisfiable ssi l existe des choix non déterministes pour lequels $M$ accepte x, i.e x in A.
\paragraph{} 
Si A est NP-diificile et $A\leq_p B$ alors B est NP difficile, car B est plus compliqué que A., pour montrer que BLOP est np complet, on montre que blop in NP et blip $\leq_p$ BLOP , comme blop plus compliqué que blip et blip np , alors blop np\\\textbf{prop} 3SAT et variante est NP-complet, \\une formule CNF avec jsuqu'a 3 littéraux par clause\\ idem mais avec exactement\\idem mais avec 3 littéraux distincts\\\\\textbf{démonstration} \\3SAT in NP : cf cours précédent\\3SAT est NP-difficile, montrons $SAT \leq_p 3SAT$ \\Exhibons F calculable en temps ploynomial telle que \\$\forall \Phi, \Phi \in SAT <=> f(\Phi) \in 3SAT$\\no doit trouver un moyen de transformer les littéraux de tailel n en littéraux de taille 3 avec des conjonctions entre eux, pour respecter CNF.\\\\F remplace chaque clause ($l_1 v l_2v...vl_k$) par la formule ($l_1vl_2v\alpha_2) \wedge (\bar{\alpha_2}vl_3n\alpha_3)\wedge...\wedge(\bar{\alpha_{k-2}}vl_{k-1}vl_k)$ où $\alpha_2,...,\alpha_{k-2}$ sont k-3 nouvelles variables, et cette formule est satisfiable ssi la clause l'est.

\section{26 Mars 2013}
\subsection{NP-complétude suite}
\paragraph{} \textbf{Lemme} $3_{\neq} SAT \leq_p 3_{ex}SAT \leq_p 3_\leq SAT$\\\textbf{Démo} Icic à chaque fois $A \leq_m B $ à prouver\\avec $A\subseteq B$ et on sait séparer A de B en temps polynomial.\\\textbf{Proposition} $3_\leq SAT \leq_p 3_{\neq} SAT$\\\textbf{Proposition} $SAT \leq_p 3_{\leq}SAT$\\\textbf{Corollaire} 3SAT est NP-Complet\\\textbf{Démonstration} Astuce : la clause ($P vQ$) ou P et Q sont des disjonctions de littéraux est satisfiable si et seulement si $(Pvx) \wedge (\bar{x} v Q)$ est satisfiable ou x n’apparaît ni dans P ni dans Q.\\\textbf{=>} Supposons (PvQ) satisfiable . Fixons une instanciaition $\sigma$ qui satisfait (PvQ). ALors on construit une instanciation $\sigma$' qui satisfait $(Pvx) \wedge (\bar{x} v Q)$  ainsi
\begin{itemize}
\item si $\sigma(P)$ ets vraie on pose $\sigma'(x)$ = faux
\item si $\sigma(P)$ est faux alors $\sigma(Q)$ ets vrai on pose $\sigma'(x)$ = vrai
\item $\forall y ,y\neq x$ , $\sigma'(y) = \sigma(y)$
\end{itemize}
\textbf{<=} Supposons $(Pvx) \wedge (\bar{x} v Q)$ satisfait par ine instanciation $\sigma$ alors $\sigma$ satisfait aussi (P $\wedge$ Q). En effet si $\sigma(X)$ ets vraie alors $\sigma(Q)$ ets vraie, et si $\sigma(X)$ ets faux alors $\sigma(P)$ ets vraie.

\paragraph{} Pour montrer  $3_{\leq}SAT \leq_p 3_{\neq}SAT$, il faut associer à toute formule $\phi$ avec jusqu'a 3 lit/clauses, une formule $f(\phi)$ avec exactement 3 lit $\neq$/clause de sorte de $\phi$ satisfiable <=> $f(\phi)$ sat.\\Il faut faire grossir les clauses à  ou  2 lit $\neq$ ON applique l'astuce une ou deux fois. Cette transformation se calcule en temps polynomial.\\Pour montrer qu e$SAT \leq_p 3_{\leq}SAT$ \\$\phi$ quelconque\\$f(\phi)$ jusqu'a 3 lit/clause\\on applique l'astuce pour faire maigrir les clauses : \\$(l_1 v l_2 v ... v l_k)$  on l'applique k-1 fois -> $(l_1 v \alpha_1) \wedge (\bar{\alpha_1} v l_2 v \alpha_2) \wedge ...\wedge (\bar{\alpha_{k-1}} v l_k)$\\cette transformation est calculable en temps polynomial donc $SAT \leq_p 3SAT$\\
\begin{verbatim}
CIRCUIT-SAT
	entrée : un circuit boolean C avec des portes NON,ET,OU, et des entrées et une sortie
	question : existe-t-il une affectation des entrées qui force la sortie de C à vrai
\end{verbatim}

\paragraph{}CIRCUIT-SAT $\in$ NP pour décoder CIRCUIT-SAT, il suffit de deviner une instanciation des entrées et d'évaluer le circuit, en temps polynomial.\\\\\textbf{Proposition} CIRCUIT-SAT est NP-Complet.\\\textbf{Démonstration} \begin{itemize}
\item[1] CIRCUIT-SAT $\in$ NP (axiome)
\item[2] $3SAT \leq_p CIRCUIT-SAT$ 
\end{itemize}
On code une formule $\phi$ de 3SAT par un circuit avec : 
\begin{itemize}
\item une entrée /variable, 
\item une porte $\neq$ /littéral négatif
\item deux portes OU par clauses
\item k-1 portes ET entre les k clauses
\end{itemize}
Cette transformation est une réduction many-one PTIME de 3SAT en CIRCUIT-SAT.

\begin{verbatim}
VERTEX-COVER
	Entrée : G=(V,E) un graphe et k appartient à N
	question : existe-t-il un sous-ensemble U \subseteq V, U de taille k tel que toute arête de G à une extrémité dans U
\end{verbatim}
\textbf{Proposition} VERTEX-COVER est NP-Complet \\Prove-it !
\begin{itemize}
\item[i] VERTEX-COVER$\in$NP : un certificat est un sous ensemble de sommets qui couvre les arêtes de G
\item[ii] Mq 3SAT $\leq_p$ VERTEX-COVER
\end{itemize}
Soit  $\phi$ une instance de 3SAT on construit $f(\phi)$ à l'aide des gadgets suivants:
\begin{itemize}
\item chaque variable x est représentés par : (x)---($\bar{x}$)
\item chaque clause ($l_1 v l_2 v l_3$) est représentée par: un triangle
\item on relie chaque sommet des gadgets de clauses aux sommets des gadgets de variables de même label
\item k = 2(n+m) avec m = \#clause et n=\#variables.
\end{itemize}
\textbf{=>} Si $\phi$ est satisfiable, on obtient une couverture de $f(\phi)$ de taille k=2m+n) ainsi : \\ on fixe une instance $\sigma$ qui satisfait $\phi$.
\begin{itemize}
\item si $\sigma(x) $ est vrai , on prend le sommet x de l'haltère x sinon le sommet $\bar{x}$
\item pour chaque clauses ($l_1 v l_2 v ... v l_n$) avec $\sigma(l_1)$ = vrai , on prend les sommet$ l_2 et l_3$
\end{itemize}
les aretes internes de chaque gadget sont couverts. une arete d'un littéral vrai est couverte par le littéral lui-meme et une arete d'un littéral faux est couverte par la clause\\\\\textbf{<= } Soit U une k-couverture de $f(\phi)$, elle couvre les haltères donc , comme elle couvre chaque haltère, U contient au moins un sommet par haltere( soit la var et la neg de la var).\\elle couvre les triangles, comme elle couvre tous les triangles, alors U contient au moins deux sommets /triangle.\\comme \#U = k =2m+n la couverture ets de la forme précédente et l'instanciation associée satisfait $\phi$
\paragraph{SUBSET-SUM}
\begin{verbatim}
SUBSET-SUM
entrée : des entiers x1,x2,...,xn de N ert un objectif S de N
question: existe-t-il un sous-ensemble des xi qui sommet à S?
peut on prendre un sous ensemble dont la somme des éléments sera S?
\end{verbatim}
\textbf{proposition} SUBSET-SUm est NP-Complet\\let's Go !!
\begin{itemize}
\item SUBSET-SUM : un certificat qui est le choix I des $x_i$ qui somment à S
\item 3SAT$\leq_p$SUBSET-SUM
\end{itemize}
soit $\phi$ une instance de 3SAT, on construit polynomialement $f(\phi)$  une instance de SUBSET-SUM qui est satisfiable ssi $\phi$ l'est.\\\\ON code $\phi$ à n variables et m clauses par des nombre $N_i$ décimaux à m+n chiffres de la manière suivantes : 
\begin{itemize}
\item on les décomposent en deux zones
\item celle de gauche code les variables,
\item celle de droite code les clauses
\item à chaque variable $x_i$ on associe un entier $N_{2i}$  et $N_{2i+1}$ qui est contruit de la manière suivante : que des zéro sauf , un 1 pour le chiffre qui correspond a $x_i$ et un 1 si $x_i$ apparaît dans la clause(ou 1 si x n’apparaît pas pour $n_{2i+1}$ qui code $\bar{x}$
\item à chaque clause $C_j$ on associe $N_{2(n+j)}$ et $N_{2(n+j)+1}$identiques et égaux à que des 0 dnas la partie variables et un 1 dans le chiffre qui code la clause
\end{itemize}
\textbf{=>} Si $\phi$ ets satisfaite par $\sigma$, il suffit de choisir $N_ {2i}$ si $\sigma(x_i)$ = VRAI, $N_{2i+1}$ sinon.\\Et $N_{2(n+j)}$ ,$N_{2(N+j)+1}$ pour compter le chiffre de $C_j$ si il vaux 1 ou 2.\\\\
\textbf{Proposition} Si $f(\phi)$ est satisfait par U. U force pour chaque $x_i$ le choix de $N_{2i}$ ou $N_{2i+1}$ fixant une instanciation $\sigma$ qui satisfait $\phi$.\\f ets calculable en temps polynomiale donc c’est une réduction \\3SAT $\leq_p$ SUBSET-SUM\\donc SUBSET-SUM est NP-Complet.\\
\end{document} 