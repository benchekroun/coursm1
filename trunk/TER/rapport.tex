\documentclass{article}
\usepackage[utf8]{inputenc} % un package
\usepackage[T1]{fontenc}      % un second package
\usepackage[francais]{babel}  % un troisième package


\title{Travaux d’Études et de Recherche\\Rapport Intermédiaire}
\author{Julien Henry\\Nicolas Lacourte-Barbadaux\\Alexandre Masson\\Léo Rousseau}
\date{14 Janvier 2013}

\begin{document}
\maketitle
\newpage
\tableofcontents
\newpage
\section{Résumé du projet}
\paragraph{} R est un logiciel de statistiques open source. De notre côté, nous développons depuis plusieurs années un logiciel de visualisation de données en java doté d'une interface graphique 3D interactive.
\paragraph{}
Il s'agit ici d'étudier le portage de cette application vers R, en se focalisant notamment sur les capacités d'interaction des librairies 3D proposées par ce système.
\newpage
\section{Introduction du domaine}
\newpage
\section{Analyse de l'existant}
\paragraph{}
R est un logiciel de statistiques open source qui fonctionne par modules
http://www.r-project.org/.
\paragraph{}
La communauté R est très active et de nombreux modules sont régulièrement proposés.
R permet facilement de charger des données, de sortir des informations statistiques, de visualiser
diverses courbes extraites des données, etc.
\paragraph{}
Par défaut, R se présente sous la forme d'une ligne de commande. Il est cependant possible de
développer des interfaces utilisateur pour contrôler l'application ainsi que pour visualiser des
résultats de traitement.
\paragraph{}
De notre côté, nous développons depuis plusieurs années un logiciel de visualisation de données
doté d'une interface graphique 3D interactive. Ce logiciel est écrit en Java.
http://www.univ-orleans.fr/lifo/software/Explorer3D/
\paragraph{}
Nous avons acquis une bonne maîtrise de la structuration d'un tel outil afin de le rendre évolutif.
Nous avons également mis en place un certain nombre de fonctionnalités interactives.
Nous étudions maintenant les possibilités d'interactions entre ce logiciel et la plateforme R, afin
d'intégrer rapidement divers outils existant sous R, mais également de diffuser notre logiciel vers
cette communauté.
\paragraph{}
Au moins deux approches sont possibles : 1/ intégrer des invocations de R depuis java. 2/ migrer
notre logiciel vers R.
\paragraph{}
Concernant le premier point, l'invocation de R depuis java est relativement simple. Elle ne sera pas
abordée dans ce TER. Concernant le second point, R propose la création de fenêtres de contrôle et
de fenêtres graphiques 2D et 3D (voir par exemple http://rgl.neoscientists.org/). Toutefois, nous ne
connaissons pas les possibilités d'interaction réelles de l'interface 3D.
\paragraph{}
L'objectif de ce TER consiste à :

• étudier le mécanisme de développement d'interfaces utilisateur sur le plan 1/ des fenêtres de
contrôle, et 2/ des fenêtres de visualisation 3D et 2D ;
• en produire une synthèse ;
• réaliser une maquette dont le contour précis sera discuté avec l'enseignant (charger un jeu de
données, visualiser en 3D après préparation des données, sélectionner des objets de la scène
3D grâce à la souris, afficher des images).

\newpage
\section{Besoins non fonctionnels}
\begin{itemize}
\item Montée en charge
\item Interaction
\item Documentation des technologies à utiliser
\end{itemize}
\subsection{Montée en charge}
\paragraph{} Comme expliqué plus haut, nous allons différer de la méthode utilisée dans Explorer3D pour toute la partie affichage. Pour savoir si les solutions choisies sont acceptables, il sera nécessaire de faire des tests de montée en charge.\paragraph{} Pour cela nous allons tester la réactivité du système aux différents événements, que ce soit des demandes de calculs, ou des interactions avec la scène 3D (rotation, zoom, déplacement, etc...).\paragraph{} Pour réaliser ces tests, nous procéderons à la main aux différentes manipulations, et nous nous baserons sur notre jugement pour percevoir la vitesse de réponse, car il n'est pas facile de chronométrer la réactivité de la scène 3D, nous allons bien entendu refaire ces test avec un nombre de plus en plus grand d'objets à afficher, pour pouvoir déterminer une limite en terme de quantité d'objets représentables.
\newpage
\section{Besoins fonctionnels}
\paragraph{} 
tout d'abord, s'agissant avant tout d'une étude, les besoins fonctionnels ne sont pas très poussés, néanmoins l'étude doit montrer la faisabilité de:\begin{itemize}
\item gestion de fenêtres de dialogue permettant le chargement de données tabulaire, le calcul de projections, l'ouverture d'une fenêtre 3D interactive.
\item fenêtre 3D interactive, permettant notamment :
\item zoom et déplacement
\item mais aussi sélection d'objet par clic
\item mise en surbrillance des objets sélectionnés
\item affichage dynamique du repositionnement spatial des objets
\item ajout au besoin d'objets supplémentaires (e.g. ellipses)
\item coexistence de plusieurs fenêtres 3D
\item mécanisme similaire à la "loupe" de Explorer3D (étudier des pistes permettant d'obtenir...)
\item AJOUT : possibilité de se mettre en écoute sur un port pour réception de commandes de type jeu de données + affichage.
\end{itemize}
\newpage
\section{Description des prototypes}
\newpage
\section{Planning, affectation des taches}
\paragraph{} Dans ce TER, il est facile de différencier deux grandes parties, car il est possible de traiter la partie concernant la scène 3D et la partie interface utilisateur séparément. Nous tiendrons compte de cette séparation pour la répartition des taches.
\newpage
\section{Bibliographie} 
\end{document}