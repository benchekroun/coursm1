\paragraph{}
Le package RGtk2 s'ajoute en tant que librairie externe pour inclure dans R la possibilité de créer des interfaces graphiques. 

\paragraph{}
Comme dans Explorer3d, on a besoin de gérer les données grâce à l'interface graphique. Cette extension de R est essentielle dans le développement de l'application. 

 \begin{center}
\includegraphics[scale=0.4]{multipleviews.png}\\
\textit{fenêtre divisé en plusieurs parties}
\end{center}

\paragraph{}
RGtk est un package dérivé de GTK2. Il est écrit en c++. GTK est un standard pour la création d'interface utilisateurs en c++ et bien d'autres langages. Grâce à RGtk2 on résout le problème de création d'interface graphique. Comme dans la capture d'écran précédente on peut créer une fenêtre avec plusieurs compartiment à l'intérieur de celle ci. ON peut aussi avoir une fenêtre pour choisir la couleur d'un objet objet affiché dans une vue RGl. 

\begin{center}
\includegraphics[scale=0.4]{colorselector2.png}\\
\textit{Sélectionneur de couleur RGtk}
\end{center}

\paragraph{}
Le but de RGtk2 est de surcoucher GTK. Avec ce package on pourra interagir avec le logiciel. \\\\\indent

Nous trouvons ce package intéressant, car il nous offre la possibilité de créer des interfaces graphiques pour l'utilisateur, ce qui lui permettra de faire les demandes de calculs, sans passer par la console. 

\begin{center}
\includegraphics[scale=0.4]{regression.jpg}\\
\textit{Un autre exemple d'interface graphique avec RGtk}
\end{center}