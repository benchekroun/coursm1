                                      
\subsection{Utiliser RGL}

\paragraph{Prérequis} Vous devez avoir une version de R récente déjà présente sur votre ordinateur ; le mieux est de récupérer les sources les plus à jour et de les compiler directement pour votre système. 
\\ \\
Le package RGL se trouve dans la liste des package disponnibles pour R à cet adresse : http://cran.r-project.org/ section "Packages", "sorted by Names", "rgl". Le fichier à récupérer est le "package source" en tar.tgz. Une fois que ce fichier est téléchargé, lancez une invite de commande R et saisissez : 

\begin{lstlisting}
> install.packages("$HOME/Downloads/rgl_0.93.928.tar.gz")
\end{lstlisting}

Le chemin passé en paramètre est un exemple, donnez le chemin vers le fichier que vous avez téléchargé. Celà devrait prendre un peu de temps. \\ \\
Une fois l'installation terminée vous devez linker le package dans l'environnement courrant (et vous devrez le faire à chaque début de session R où vous aurez besoin de RGL) : 

\begin{lstlisting}
> library(rgl)
\end{lstlisting}
\newpage
\subsection{Première Session d'affichage 3D en R}

Voici un code court qui va afficher N sphères de tailles, coordonnées et rayons aléatoires : 

\begin{lstlisting}
premier.test <- function(N)
{
  rgl.open()
  rgl.bg(color="black")
  rgl.spheres(x=rnorm(N), y=rnorm(N), z=rnorm(N), 
              radius=runif(N), color=rainbow(N))
}

> premier.test(10)
\end{lstlisting}

Maintenant on va expliquer chaque lignes de ce script. La première commande, \textbf{rgl.open()} va permettre d'ouvrir une feneêtre openGL. Pour l'instant cette fenêtre est vide, on n'a pas encore demandé d'affichages. Ensuite la seconde ligne \textbf{rgl.bg(color="black")} va permetre de changer la couleur d'arrière plan. 
\\ \\
Enfin, la dernière commande, la plus importante va demander au moteur OpenGL d'afficher N sphères. \textbf{rgl.spheres(x=rnorm(N), y=rnorm(N), z=rnorm(N), radius=runif(N), color=rainbow(N))}. Nous allons l'expliquer pour chaque paramètres.
\begin{itemize}
\item \textit{x,y,z} : Trois vecteurs de taille identique représentant pour chaque sphère les coordonnées de son centre
\item \textit{radius} : Un vecteur de la même taille que x,y et z qui contient pour chaque sphère son rayon
\item \textit{color} : Un vecteur de la même taille que x,y et z contenant les couleurs pour chaque sphères (ou alors une seule couleur pour toute les sphères)
\end{itemize}

\newpage
\subsection{Primitives de dessin}

\subsubsection{Points et Lignes}
Dessiner un point ou un ensemble de points. Le paramètre \textit{color} est disponnible pour l'ensemble des primitives
\begin{lstlisting}
> rgl.open()
> rgl.bg(color="black")
> rgl.points(x, y, z, color="red" )
> rgl.lines(x, y, z)
\end{lstlisting}

\begin{itemize}
\item \textit{x,y,z} : Trois vecteurs de taille identique représentant pour chaque point les coordonnées de son centre
\item \textit{color} : Un vecteur de la même taille que x,y et z contenant les couleurs pour chaque points (ou alors une seule couleur pour tous les points)
\end{itemize}



\subsubsection{triangles}

Dessiner un ensemble de triangles

\begin{lstlisting}
> rgl.open()
> rgl.bg(color="black")
> rgl.triangles(x, y, z, normals=NULL, texcoord=NULL )
\end{lstlisting}

\begin{itemize}
\item \textit{x,y,z} : Trois vecteurs de taille identique représentant pour chaque point les coordonnées de son centre
\item \textit{Normals} : Un vecteur contenant pour chaque sommet du reseau de triangle
\item \textit{texcoords} : Un vecteur contenant pour chaque sommet du reseau de triangle ses coordonnées de texture.
\end{itemize}

\subsubsection{Spheres}

Dessiner une sphère ou un ensemble de sphères. Le paramètre \textit{color} est disponnible pour l'ensemble des primitives
\begin{lstlisting}
> rgl.points(x, y, z,r color="red" )
\end{lstlisting}

\begin{itemize}
\item \textit{x,y,z} : Trois vecteurs de taille identique représentant pour chaque point les coordonnées de son centre
\item \textit{r} : Un vecteur de la même taille que x, y et z contenant le rayon pour chaque sphères 
\item \textit{color} : Un vecteur de la même taille que x,y et z contenant les couleurs pour chaque sphères(ou alors une seule couleur pour toutes les sphères)
\end{itemize}

\newpage
\subsection{recapitulatif rapide des fonctions de base de RGL}
\textit{gestion de l'outils}\\\\
\begin{tabular}{|c|c|}
\hline 
rgl.open()/open3d() & Ouvre une nouvelle fenêtre \\ \hline
rgl.close() & Ferme la fenêtre actuelle \\ \hline
rgl.cur() & Retourne le nombre de fenêtres ouvertes \\ \hline
rgl.set(idFenetre) & Selectionne la fenêtre corespondante comme active \\ \hline
rgl.quit() & Ferme toute les fenêtres ouvertes et décharge la console courante de RGL \\ \hline
\end{tabular}\\

\textit{gestion de la scène }\\\\
\begin{tabular}{|c|c|}
\hline
rgl.clear() & Nettoie la scène en fonction du paramètre passé (ex "shapes" ou "lights") \\ \hline
rgl.pop(type="shapes") & Supprime le dernier objet ajouté à la scène \\ \hline

\end{tabular}\\

\textit{fonctions d'export}\\\\
\begin{tabular}{|c|c|} 
\hline
rgl.snapshot(nameFile) & Sauvegarde un screenshot de la scène courante au format PNG \\ \hline

\end{tabular}\\

\textit{gestion des objets}\\\\
\begin{tabular}{|c|c|} 
\hline
rgl.points(x,y,z,..)/points3d() & Dessine un point aux coordonnées x,y et z (voir plus haut) \\ \hline
rgl.lines(x,y,z,..) & Dessine une/des lignes sur l'axe (voir détails plus haut) \\ \hline
rgl.triangles(x,y,z,..) & Dessine un/des triangles (voir détails plus haut) \\ \hline
rgl.quads(x,y,z,..) & Dessine un/des carrés \\ \hline
rgl.spheres(x,y,z,r,..) & Dessine une/des sphères (voir détails plus haut) \\ \hline
rgl.texts(x,y,z,text,..) & Ajoute une texture à la scène \\ \hline
rgl.surface(x,y,z) & Ajoute a la surface .. \\ \hline

\end{tabular}\\
\newpage
\textit{gestion de l'environnement}\\\\
\begin{tabular}{|c|c|}
\hline
rgl.viewpoint(theta,phi,fov,zoom,interactive) & Définie le point de vue (theta, phi)\\ &  dans des coordonnées polaires  \\ & avec un angle de vue fov et un facteur zoom.\\ & Le marqueur logique interactive définie \\ & si ou non la navigation est autorisée \\ \hline
rgl.light(theta,phi,..) & Ajoute une source de lumière à la scène \\ \hline
rgl.bg(..) & Attribue le background de la scène \\ \hline
rgl.bbox(..)/bbox3d & Attribue un support visuel dynamique aux objets\\ & ainsi qu'une échelle visible de leurs coordonnées \\ \hline

\end{tabular}\\
