\section{Introduction du domaine}
\paragraph{La classification}

La classification est une forme d’apprentissage automatique. Elle consiste,
étant donné un ensemble d’éléments, a les répartir dans différentes catégories. Il existe
deux courants majoritaires de classification.

\paragraph{La classification supervisée}


Dans la classification supervisée on cherche à déterminer la
étant-donné un ensemble d’apprentissage dont on connait déjà la classification. Par
exemple si on classe des animaux on doit pouvoir déterminer que le cobra est un
reptile parce qu’il a une description proche des reptiles déjà présents dans l’ensemble
initial.

\paragraph{La classification non-supervisée}

La classification non-supervisée, ou encore clustering, a un but différent. On
cherche a classer un ensemble de données en plusieurs classes sans les connaître a
l’avance. On calcule une répartition pour l’ensemble. C’est ce que fait par exemple
l’algorithme des K-Moyennes[1].



\newpage